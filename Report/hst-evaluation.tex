\chapter{Evaluatie}
\label{hoofdstuk:evaluatie}
Dit hoofdstuk behandelt enkele gebruikte methodes om de ontwikkelde systemen te evalueren. Twee van deze methodes vinden hun oorsprong in de computervertaling. Ze zijn ontwikkeld om automatisch de kwaliteit van vertalingen te meten. Het doel van deze evaluatiealgoritmes is om zo goed als mogelijk menselijke evaluatiescores te benaderen. De laatste vorm van evaluatie kijkt naar statistieken van de gegenereerde zinnen.


\section{Automatische evaluatie}
Ideaal gezien beoordelen meerdere mensen elke zin om tot een betrouwbare evaluatie te komen. Deze beoordeling kan bijvoorbeeld door een score te geven op de kwaliteit van elke zin. Een andere optie is om proefpersonen te laten oordelen of een automatisch gegenereerde zin beter, hetzelfde of slechter is dan de overeenkomstige referentiezin. Het nadeel van elk van deze methodes is de nood aan meerdere proefpersonen die voor elke gebruikte methode of instelling van de netwerken, alle duizend zinnen van de validatieverzameling of de testverzameling moet beoordelen. Het is duidelijk dat dit een kostelijke operatie is. Daarvoor bieden automatische evaluatiealgoritmes een oplossing.

In wat volgt bespreken we eerst twee algoritmes uit de computervertaling namelijk Bleu en METEOR. \todo{bleu en meteor} Deze algoritmes zijn bruikbaar omdat de gegenereerde zinnen kunnen worden beschouwd als "vertalingen" van de afbeeldingen. Het is aangetoond dat de twee methodes correleren met menselijke evaluaties van gegenereerde zinnen. Van de gebruikte methodes heeft METEOR de hoogste correlatie. \todo{referentie elliot en keller}
Geen van beide algoritmes is echter perfect voor het meten van de kwaliteit.
Om die reden bekijken we ook nog enkele statistieken van de gegenereerde zinnen om hieruit andere interessante fenomenen af te leiden.

\section{Bleu}
Kort gesteld berekent Bleu scores van computervertalingen op basis van overeenkomst van woorden of aangesloten woorden met de referentiezinnen. Verschillende vormen van Bleu kunnen worden gebruikt naargelang het aantal n-grams. Het gebruik van Bleu heeft echter wel enkele nadelen.

\subsection{Algoritme}
Om de n-gram Bleu-score van een zin te berekenen bepaalt het algoritme eerst de \textit{modified n-gram precision}. Dit is een vorm van precisie die rekening houdt met het aantal keer dat elk woord in de referentiezinnen voorkomt. Op deze manier krijgen zinnen als \texttt{the the the the the} een lage score omdat \texttt{the} nooit vijfmaal voorkomt in een referentiezin. De berekening van \textit{modified n-gram precision} gebeurt door eerst het maximum te nemen van het aantal keer dat een specifieke woordsequentie voorkomt in elke referentiezin. Vervolgens telt het algoritme het aantal voorkomens van deze sequentie in de gegenereerde zin. Het minimum van deze twee getallen ($Count_ {clip}$) wordt dan voor elk woord in de vertaalde zin opgeteld en gedeeld door het totaal aantal woorden in de gegenereerde zin.

\begin{equation}
p_{modified}(s) =
\frac{\sum\limits_{ngram \in s} Count_{clip}(ngram)}{\sum\limits_{ngram' \in s} Count(ngram')}
\label{formule:ngramprecision}
\end{equation}
Vanuit deze formule volgt een score voor een volledige corpus van gegenereerde zinnen als volgt:
\begin{equation}
p_{n} =
\frac{\sum\limits_{C \in \{Candidates\} } \sum\limits_{ngram \in C} Count_{clip}(ngram)}{\sum\limits_{C' \in \{Candidates\} } \sum\limits_{ngram' \in C'} Count(ngram')}
\label{formule:corpus_modified}
\end{equation}

Vanuit deze scores voor $i=1$ tot en met $n$ is het mogelijk om een n-gram Bleu-score te bepalen. Dit gebeurt door het gemiddelde logaritme te nemen met uniforme gewichten, wat overeenkomt met het geometrisch gemiddelde van de modified n-gram precisions.
Deze score dwingt echter niet de juiste lengte van de zin. Om die reden bepaalt de paper een extra multiplicatieve factor namelijk de Brevity Penalty ($BP$). Voor elke gegenereerde zin bepaalt het algoritme de referentiezin met de dichtstbijzijnde lengte. De lengte daarvan noemt de paper de "beste match lengte". Vervolgens worden zowel de lengtes van de genereerde zinnen als de beste match lengte opgeteld respectievelijk $c$ en $r$. Berekenen van de Brevity Penalty kan met volgende formule:
\begin{equation}BP=
 \begin{cases}
1 & if c > r \\
e^{1-r/c} & else
\end{cases}
\end{equation}

De uiteindelijke Bleu-N score is dan gelijk aan:
\begin{equation}
BLEU = BP\cdot exp(\sum\limits_{n=1}^N w_nlog(p_n))
\end{equation}
Hierbij is $w_n$ gelijk aan $1/N$ wanneer het uniforme geometrisch gemiddelde wordt genomen.

\subsection{Nadelen en gebruik}
De BLEU-score is de meest gebruikte gebruikte evaluatiemethode in de literatuur. De meeste papers die afbeeldingsbeschrijvingen genereren bespreken Bleu-1 tot Bleu-4 scores. De literatuur lijkt het echter niet eens over het gebruik van de Brevity Penalty. Sommige papers vermelden expliciet dat ze het niet gebruiken, maar andere volgen de paper van Papinemi volledig. Door dit verschil in evaluatie is het vergelijken van systemen niet altijd eerlijk. Wanneer de gegenereerde zinnen lang genoeg zijn komen de scores wel overeen.
In onze experimenten stellen we de Brevity Penalty steeds gelijk aan 1.

In \todo{cite elliott} toont Elliot et al. aan dat BLEU slechts een matige correlatie heeft met menselijke beoordeling.
Hiervoor zijn meerdere verklaringen mogelijk. Zo kijkt BLEU enkel naar exacte woordmatches maar houdt het geen rekening mee met semantisch gelijkaardige woorden. Zo geeft bijvoorbeeld het gebruik van een synoniem geen hogere score terwijl dit bij menselijke evaluaties geen probleem is. Ook werkwoorden die semantische informatie bevatten die in de referentiezin als substantief voorkomen geven problemen. Voorbeelden van zulke situaties:
\\

Reference 1: \texttt{Two boys are on their bikes.}

Candiate 2: \texttt{Two boys are on their bicycles.}

Reference 2: \texttt{A man is skiing down a hill.}

Candidate 2: \texttt{A man is going down a hill on his skis}
\\

METEOR is daarom een evaluatiealgoritme met als doel om foute evaluaties als deze te vermijden.

\section{Meteor}
Elliott et al. \todo{cite} toonden aan dat zeker unigram Bleu (B1) slechts een zwakke correlatie heeft met menselijke evaluatie. Hogere orde Bleu-scores hebben slechts een matige correlatie. METEOR is een metriek die specifiek ontworpen is om de tekortkomingen van Bleu te verbeteren. Van de door Elliott onderzochte evaluatiemechanismen, correleert METEOR het beste met menselijke evaluatie.

\subsection{Algoritme}
Meteor \todo{cite meteor!} evalueert vertaalde zinnen door ze te aligneren met de referenties en hiervan per zin een score te berekenen. Bij het berekenen van de score wordt net zoals bij Bleu gekeken naar de precisie. Daarnaast heeft in contrast met Bleu ook de recall een invloed. De paper bevat een implementatie in de vorm van een JAR-bestand, zodat er geen twijfel bestaat over het gebruik van het algoritme.

Concreet probeert het algoritme twee zinnen te aligneren met behulp van vier matchers. Als eerste is er een match wanneer de woordvormen exact dezelfde zijn. Als tweede is er een match wanneer met een Snowball Stemmer (\todo{cite porter 2001} porter) 'gestemde' woorden gelijk zijn. Een derde matcher kijkt naar overeenkomsten in de WordNet synoniemenlijst van elk woord. \todo{cite miller en felbaum} Als laatste vormen frases een match wanneer ze in zogenoemde frasetabellen voorkomen.
Elke matcher heeft een bepaald gewicht. In de experimenten gebruiken we de standaardgewichten 0.85, 0.2 ,0.6 en 0.75.

Uiteindelijk worden alle matches gegeneraliseerd tot frase matches met een bepaalde startpositie en eindlengte. Een van de doel van de Meteor score is om zoveel mogelijk woorden af te dekken in de twee zinnen. Daarnaast moet het aantal \textit{chunks} minimaal zijn. Denkowski et al. defini\"eren een chunk als aaneengesloten en identiek geordende matches tussen de twee zinnen. De uiteindelijke Meteor score bestaat dan uit de F-score vermenigvuldigd met een penalisatiefactor op basis van het aantal chunks.

\begin{equation}
Score = (1 - Pen)*F_{mean}
\end{equation} 
\begin{equation}
Pen = \gamma (\frac{ch}{m})^\theta
\end{equation}
\begin{equation}
F_{mean} = \frac{PR}{\alpha P + (1- \alpha)R}
\end{equation}

Hierbij zijn $P$ en $R$ respectievelijk de gewogen precisie en recall van de gealigneerde unigrams ($m$) tussen kandidaat- en referentiezin.

Wanneer er meerdere referenties zijn, bepaalt het maximum van de individuele scores voor elke referentie de score van de vertaling.

%%% Local Variables:
%%% mode: latex
%%% TeX-master: "masterproef"
%%% End:
