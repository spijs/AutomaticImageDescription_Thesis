\chapter{Inleiding}
\label{inleiding}
%In dit hoofdstuk wordt het werk ingeleid. Het doel wordt gedefinieerd en er
%wordt uitgelegd wat de te volgen weg is (beter bekend als de rode draad).

Het concrete doel van deze masterproef is het maken van een systeem dat in staat is om afbeeldingen te beschrijven. Hierbij komen twee domeinen van de computerwetenschappen samen, enerzijds computervisie en anderzijds natuurlijke taalverwerking. Concreet moet het geleerde model ongeziene afbeeldingen kunnen omzetten tot vloeiende, grammaticaal correcte Engelstalige zinnen. Bovendien moeten deze zinnen de afbeelding zo volledig mogelijk beschrijven. Het eerste hoofdstuk van deze masterproef werkt deze doelstelling concreet uit.


Om dit probleem aan te pakken volgt eerst een uitgebreide literatuurstudie. Deze studie bekijkt \textbf{enkele} relevante werken uit de computervisie en uit de natuurlijke taalverwerking. Daarnaast ligt de focus voornamelijk op recente papers die een soortgelijk doel als deze masterproef nastreven. Hieruit volgt een vergelijking van hoe deze papers afbeeldingen en zinnen voorstellen. Vervolgens onderzoekt de literatuurstudie op welke verschillende manieren deze representaties leiden tot een concreet systeem voor afbeeldingsbeschrijving.

Na de literatuurstudie volgt een theoretische uitdieping van de ervoor beschreven concepten. Neurale netwerken vormen de basis van het uiteindelijke systeem. Om die reden is hier een sectie aan gewijd. Daarna komt een uitdieping van statistische concepten die nuttig kunnen zijn voor het beschrijven van afbeeldingen.

Het volgende hoofdstuk bespreekt de gebruikte methodologie.
E\'en paper uit de literatuurstudie vormt het basiswerk. Deze paper gaat gepaard met een vrij te gebruiken implementatie. Deze implementatie is dan ook het startpunt van de masterproef. Het hoofdstuk begint met een uitgebreide bespreking van dit startpunt.
Daarna volgt een uiteenzetting van de uitbreidingen op deze implementatie. Deze uitbreidingen bevatten nieuwe datasets, andere types van neurale netwerken, verschillende vormen van extra semantische informatie en een vorm van normalisatie bij het genereren van de zinnen.

Om de verschillende uitbreidingen te vergelijken met het startpunt en modellen uit de literatuur, moet er een manier zijn om te evalueren. Menselijke evaluatie is het ideaal, maar is niet altijd haalbaar. Om die reden bestaan er verschillende methodes om een getraind systeem te beoordelen. Dit hoofdstuk biedt een overzicht van automatische evaluatiemethodes uit de literatuur.

Het volgende hoofdstuk bespreekt de uitgevoerde experimenten in detail. Dit bevat onder andere de configuraties van de netwerken en hun uitbreidingen. Daarnaast bespreekt het een tweede type van experiment, dat de ruisgevoeligheid van twee systemen nagaat.

Daarna volgt een bespreking van de resultaten van de voorgaande experimenten en een vergelijking met de beste werken uit de literatuur.

We besluiten onze bijdrage met de conclusies die gemaakt zijn doorheen het proces van de masterproef.
 
%%% Local Variables:
%%% mode: latex
%%% TeX-master: "masterproef"
%%% End:
