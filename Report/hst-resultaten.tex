\chapter{Resultaten} % (fold)
\label{cha:resultaten}
Dit hoofdstuk bevat een overzicht van de verschillende experimenten die zijn uitgevoerd. Eerst beschrijven we de resultaten van onze eigen implementaties. Daarna volgt een kritische vergelijking met de state-of-the art resultaten. Een laatste set van experimenten gaat over een vergelijking van CCA en LDA als gids voor de gLSTM implementatie. Meer specifiek focussen we op hoe deze twee presteren bij aanwezigheid van ruis in de trainingsdata.

\section{Eigen implementaties} % (fold)
\label{sec:eigen_implementaties}

% section eigen_implementaties (end)

\section{Vergelijking met literatuur} % (fold)
\label{sec:vergelijking_met_literatuur}

\begin{table}
	\begin{tabular}{llllll}
		~                     & B1   & B2   & B3   & B4   & Meteor \\ \hline
		Google NIC\cite{Google}            & 66.3 & 42.3 & 27.7 & 18.3 & ~      \\
		gLSTM CCA + gauss\cite{Fernando2015}     & 64.6 & 44.6 & 30.5 & 20.6 & 17.91  \\
		gLSTM CCA + polyn.\cite{Fernando2015}    & 59.8 & 41.3 & 29.3 & 19.2 & 18.58  \\
		Xu attention\cite{Xu2015}         & 66.9 & 43.9 & 29.6 & 19.9 & 18.46  \\
		Karpathy\cite{Karpathy2015}              & 57.3 & 36.9 & 24   & 15.7 & ~      \\
		Attention+scenevector\cite{Jin2015} & \textbf{67}   & \textbf{47.5} & \textbf{33}   & \textbf{24.3} & \textbf{19.4}   \\
	\end{tabular}
	\label{table:results_literature}
	\caption{Vergelijking van state-of-the-art resultaten met onze resultaten}
\end{table}
% section vergelijking_met_literatuur (end)

\section{Ruisgevoeligheid van CCA en LDA} % (fold)
\label{sec:ruisgevoeligheid_van_cca_en_lda}

% section ruisgevoeligheid_van_cca_en_lda (end)

\section{Besluit} % (fold)
\label{sec:besluit}

% section besluit (end)

