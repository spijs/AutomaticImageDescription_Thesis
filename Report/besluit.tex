\chapter{Besluit}
\label{besluit}
Deze masterproef probeert twee bestaande systemen voor afbeeldingsbeschrijving te verbeteren. Deze systemen gebruiken een convolutioneel netwerk om een afbeelding om te zetten tot een vectorvoorstelling. Deze voorstelling dient dan als input van een recurrent neuraal netwerk dat als taalmodel dient. Samen met een beam-search-algoritme is dit taalmodel in staat om beschrijvingen te genereren.

De verbeteringen op de bestaande systemen gebeurt op drie manieren. Ten eerste is er het afleiden van extra informatie uit een afbeelding, hetzij door het extraheren van onderwerpen, hetzij door een multimodale projectie. Beide gebruikte methodes zijn variabel in het aantal gebruikte dimensies. Deze masterproef bestudeert dan ook de invloed van de dimensionaliteit van de extra informatie. 
De recente dataset Flickr30kEntities vormt een derde bron van informatie, maar blijkt in deze opstelling te groot om mee te werken.
Een tweede aangebrachte verbetering is het normaliseren van de zinnen tijdens het generatieproces. Dit kan leiden tot zinnen die meer informatie bevatten of tot een betere verdeling van de zinslengtes. Tenslotte bestudeert deze masterproef de mogelijke invloed van een aantal parameters van het generatieproces.

De automatische evaluatiemethodes Bleu en Meteor uit de machinevertaling maken een objectieve vergelijking van de verschillende systemen mogelijk. Daarnaast bieden statistieken over woordgebruik, zinslengte en uniciteit verdere inzichten.

Deze thesis experimenteert ook met de robuustheid van de twee manieren om semantische informatie toe te voegen. Een aantal experimenten proberen de invloed van ruis op deze informatie te bepalen.

Dit hoofdstuk biedt eerst een overzicht van de belangrijkste resultaten en de bekomen verbeteringen. Daarna volgt een overzicht van de mogelijke uitbreidingen.
\section{Resultaten}
\subsection{Toevoegen van semantische informatie}
Het gebruik van extra semantische informatie leidt tot verbeteringen in de kwaliteit van de gegenereerde zinnen. 
Bij RNN zorgt het gebruik van de onderwerpen afgeleid uit de afbeeldingen voor een hogere score ten opzichte van het referentiemodel. 
De veronderstelling dat de onderwerpen de gegenereerde zinnen beter doet aansluiten bij de afbeelding is dus correct.
Bij LSTM geeft zowel de afgeleide onderwerpen als de multimodale projectie een hogere score dan het referentiemodel. 
De scores van beide technieken om extra informatie toe te voegen liggen bij LSTM zeer dicht bij elkaar, maar het gebruik van onderwerpen scoort lichtjes beter op de metrieken die het dichtst aanleunen bij menselijke evaluatie.

\subsection{Normalisatie}
Om de lengte van de gegenereerde zinnen beter te doen aansluiten bij die van de trainingsverzameling implementeert deze masterproef een Gaussfunctie. Hierdoor krijgen zinnen die te hard afwijken van de lengteverdeling uit de trainingsset een lagere score. 
Deze methode heeft een zeer grote invloed op de gegenereerde zinnen. De gemiddelde lengte stijgt bij RNN van 7 naar 10, en bij LSTM van 8 naar 10.

Een tweede vorm van normalisatie probeert de gegenereerde zinnen informatiever te maken. Alle woorden uit de woordenschat krijgen een score op basis van hoeveel keer ze voorkomen in de trainingsverzameling. Woorden die minder voorkomen zijn specifieker en krijgen een hogere score. Op basis van deze scores krijgen de woorden in de gegenereerde zinnen een aangepaste gewicht. Deze normalisatie leidt effectief tot zinnen die meer veelzeggende woorden bevatten. Dikwijls leidt dit ook tot een zin van hogere kwaliteit, maar in een aantal gevallen voegt het systeem schijnbaar willekeurig een aantal woorden met een hoge score toe die weinig met de afbeelding te maken hebben. De evaluatiemetrieken oordelen ook dat in zijn geheel de zinnen met deze normalisatie slechter presteren, maar voor een mens zijn een groot aantal van de zinnen van betere kwaliteit.

\subsection{Invloed van parameters}
Verschillende parameters van het systeem hebben een invloed op de resultaten. Ten eerste is er de grootte van de beam-search bij het zoeken van de beste beschrijvingen. Naarmate de grootte stijgt zijn de scores beter, tot aan een plafond. Afhankelijk van het gebruikte systeem ligt de optimale grootte tussen 25 en 75. Ook de dimensionaliteit van de vector met extra semantische informatie heeft een invloed. Bij het gebruik van onderwerpverdelingen is de invloed variabel. Bij langere zinnen, bijvoorbeeld door Gauss-normalisatie, is het beter om een groter aantal onderwerpen te kiezen. Voor kortere zinnen verzwakt dit effect. Bij multimodale projectie is dit fenomeen niet zichtbaar. De modellen die gebruik maken van 256 dimensies scoren wel beter dan die met 128 en 512 dimensies. Bij 128 dimensies is de informatie wellicht te weinig. Bij 512 daarentegen komen veel onbelangrijke elementen in de vector die doorwegen in het uiteindelijke resultaat.

\subsection{Robuustheid van semantische informatie}
Beide methodes om semantische informatie toe te voegen aan de bestudeerde systemen ondervinden nadeel van ruis op de trainingsdata. De absolute en relatieve achteruitgang is kleiner bij het gebruik van een multimodale projectie. Dit komt vermoedelijk omdat het netwerk dan slechter in staat is om het verband te leren tussen de trainingszinnen en de vector met extra informatie.

\section{Toekomstig werk}
Zoals beschreven in het literatuuroverzicht en de resultaten scoren de modellen die gebruik maken van aandachtsmechanismes het hoogste. Het gebruik van aandachtsvectoren is echter te complex voor deze masterproef. Het is zeker interessant om te experimenteren met het toevoegen van aandachtsinformatie aan de systemen voorgesteld in deze masterproef om tot grotere verbeteringen te komen. 

Een ander mogelijk vervolg op deze masterproef gaat dieper in op de robuustheid van de verschillende systemen om semantische informatie toe te voegen aan de generatiesystemen.
De aanpak in de experimenten is vrij rudimentair, dus het zou zeker lonen om te onderzoeken wat de invloed is van verschillende gradaties van perturbatie, alsook verschillende types van ruis.
%%% Local Variables: 
%%% mode: latex
%%% TeX-master: "masterproef"
%%% End: 
