\documentclass[master=cws,masteroption=ai]{kulemt}
\setup{title={Describing Images with Natural Language},
  author={Thijs Dieltjens \\ Wout Vekemans},
  promotor={Prof.\,dr.\,ir.\ Marie-Francine Moens},
  assessor={Ir.\,W. Eetveel},
  acyear={2015 -- 2016},
  assistant={Ir.\ S. Zoghbi}}
% De volgende \setup mag verwijderd worden als geen fiche gewenst is.
\setup{filingcard,
  translatedtitle={The best master thesis ever},
  udc=621.3,
  shortabstract={Hier komt een heel bondig abstract van hooguit 500
    woorden. \LaTeX\ commando's mogen hier gebruikt worden. Blanco lijnen
    (of het commando \texttt{\string\pa r}) zijn wel niet toegelaten!
    \endgraf \lipsum[2]}}
% Verwijder de "%" op de volgende lijn als je de kaft wil afdrukken
% \setup{coverpageonly}
% Verwijder de "%" op de volgende lijn als je enkel de eerste pagina's wil
% afdrukken en de rest bv. via Word aanmaken.
%\setup{frontpagesonly}

% Kies de fonts voor de gewone tekst, bv. Latin Modern
\setup{font=lm}

% Hier kun je dan nog andere pakketten laden of eigen definities voorzien

\usepackage{url}
\usepackage{todonotes}
\usepackage{tikz}
% Tenslotte wordt hyperref gebruikt voor pdf bestanden.
% Dit mag verwijderd worden voor de af te drukken versie.
\usepackage[pdfusetitle,colorlinks,plainpages=false]{hyperref}

\newcommand{\myvector}[1]{$\mathbf{#1}$}

%%%%%%%
% Om wat tekst te genereren wordt hier het lipsum pakket gebruikt.
% Bij een echte masterproef heb je dit natuurlijk nooit nodig!
\IfFileExists{lipsum.sty}%
 {\usepackage{lipsum}\setlipsumdefault{11-13}}%
 {\newcommand{\lipsum}[1][11-13]{\par Hier komt wat tekst: lipsum ##1.\par}}
%%%%%%%

%\includeonly{hfdst-n}
\begin{document}

\begin{preface}
  Dit is mijn dankwoord om iedereen te danken die mij bezig gehouden heeft.
  Hierbij dank ik mijn promotor, mijn begeleider en de voltallige jury.
  Ook mijn familie heeft mij erg gesteund natuurlijk.
\end{preface}

\tableofcontents*

\begin{abstract}
  In dit \texttt{abstract} environment wordt een al dan niet uitgebreide
  samenvatting van het werk gegeven. De bedoeling is wel dat dit tot
  1~bladzijde beperkt blijft.

  \lipsum[1]
\end{abstract}                                                                                                                                                                                                                                                                                                                                                                                                                                                                                                                                                                                                                                                                                                                 

% Een lijst van figuren en tabellen is optioneel
%\listoffigures
%\listoftables
% Bij een beperkt aantal figuren en tabellen gebruik je liever het volgende:
\listoffiguresandtables
% De lijst van symbolen is eveneens optioneel.
% Deze lijst moet wel manueel aangemaakt worden, bv. als volgt:
\chapter{Lijst van afkortingen en symbolen}
\section*{Afkortingen}
\begin{flushleft}
  \renewcommand{\arraystretch}{1.1}
  \begin{tabularx}{\textwidth}{@{}p{12mm}X@{}}
    LDA  & Latent Dirichlet Allocation \\
    RNN & Recurrent Neuraal Netwerk \\
    CNN & Convolutioneel Neuraal Netwerk \\
    LSTM & Long Short Term Memory \\
    CCA & Canonical Correlation Analysis \\
    RFF & Random Fourier Feature
    
    
  \end{tabularx}
\end{flushleft}


\section*{Symbolen}
\begin{flushleft}
  \renewcommand{\arraystretch}{1.1}
  \begin{tabularx}{\textwidth}{@{}p{12mm}X@{}}
    $\theta$   & Kansverdeling van onderwerpen per document (LDA) \\
    $\phi$   & Kansverdeling van woorden per onderwerp (LDA)  \\
  \end{tabularx}
\end{flushleft}

% Nu begint de eigenlijke tekst
\mainmatter

\include{inleiding}
\chapter{Gerelateerd Werk}
\label{hoofdstuk:related}
Het automatisch genereren van captions voor ongeziene afbeeldingen is een complex proces. Het combineert zowel computervisie (vision??) (CV) als natuurlijke taalverwerking (NLP) \todo{moet dees in ne lijst van afkortingen ofzo? }. Vele modellen zijn al voorgesteld die telkens elementen uit beide onderzoeksvelden combineren om tot een oplossing te komen. Daarom volgt er een opdeling van de gerelateerde literatuur gemaakt op basis van de gebruikte technieken in deze twee domeinen. 

Het genereren van captions kan worden beschouwd als een description retrieval probleem.\cite{Hodosh2013} \todo{Juist referentie} Om die reden bevat het volgende overzicht ook methodes van werk dat zich enkel focust op het opvragen van een zo goed mogelijke bestaande caption.

\section{Afbeelding representaties}
Alle huidige modellen gebruiken technieken uit computer vision om nuttige features af te leiden uit afbeeldingen. Nuttige features omvatten onder andere het detecteren van acties, scenes en objecten en hun attributen en relaties. \cite{Bernardi} \todo{Juiste referentie fixen.} Deze features vormen dan de basis voor een representatie van de afbeelding die als input dient voor het generatie (of retrieval) model. 

Eerst volgt een bespreking van technieken die in het verleden zijn gebruikt. Daarna volgt de techniek die in de meer recente literatuur voorkomt namelijk Convolutional Neural Networks (CNN).

Alle huidige modellen gebruiken detectietechnieken uit computer vision om nuttige features af te leiden uit afbeeldingen. Deze features bevatten onder andere gedetecteerde acties, sc\`enes en objecten en hun attributen en relaties. Deze features vormen de basis voor een representatie van de afbeelding die als input dient voor een generatie (of retrieval) model. 

Eerst volgt een bespreking van technieken die in het verleden gebruikt zijn. Daarna volgt een beschrijving van Convolutional Neural Networks, een techniek die in de meer recente literatuur veelvuldig aan bod komt.

\subsection{Oorpsronkelijke CV modellen}
De literatuur gebruikt meerdere technieken uit computer vision om nuttige features af te leiden uit afbeeldingen. Features die in de vroegste papers over dit onderwerp voorkomen zijn bijvoorbeeld scene classificaties, object detecties en attribuut classifiers \cite{Farhadi2010},\cite{Yang2011}\cite{Patterson} \todo{fix cite}.Hiervoor gebruiken ze bestaande classifiers en detectors zoals \cite{Felzenszwalb2008}, Im2Text \cite{Ordonez2011} en GIST\cite{Oliva2006}. \todo{fix cite}Ee\'n of meerdere van deze features kan dan rechtstreeks de representatie vormen van een afbeelding. \cite{Farhadi2010} \cite{Yang2011} \cite{Li} \cite{Mitchell} gebruiken de features echter als input voor het vormen van abstracte afbeeldingsrepresentaties in de vorm van tupels. Deze tupels bevatten dan objecten, acties tussen objecten, sc\`ene types en/of spatiale relaties.

Een andere manier om afbeeldingen te representeren zijn Visual Dependency Representations (VDR) zoals voorgesteld door \cite{Eliott2013}. VDR's gebruiken een dependency graaf om de spatiale relaties tussen objecten voor te stellen. VDR's kunnen geleerd worden op basis van geannoteerde training data of automatisch met behulp van objectherkenning \cite{Eliott2015} of de labels in abstracte scenes. \cite{Gilberto2015} 

\subsection{CV met behulp van neurale netwerken}
Voor de meeste taken blijkt echter dat \todo{dat CNN beter presteren dan bovenstaande ... klinkt positiever vind ik} bovenstaande methodes minder goed presteren dan convolutionele neurale netwerken (CNN). Deze CNNs zijn deep learning neurale netwerken met tot 15 verborgen lagen. Convolutionele neurale netwerken hebben minder verbindingen en parameters dan overeenkomstige feedforward neurale netwerken terwijl ze niet veel slechter presteren. \cite{Krizhevsky2012}\todo{cite fix} De meest recente publicaties maken hier dan ook gebruik van. 

De gebruikte CNN's zijn netwerken die getraind zijn op ImageNet.\cite{Krizhevsky2012}\todo{fix cite} ImageNet is een dataset bestaande uit miljoenen afbeeldingen die gelabeled zijn binnen enkele duizenden categorie\"en. Het netwerk leert afbeeldingen correct te labelen. De juiste aantallen hangen af van het jaar van de gebruikte ImageNet classification challenge data. Vaak gebruikte CNN modellen zijn onder andere AlexNet \cite{Krizhevsky}\todo{fix} en het recentere VGGNet\cite{Simonyan}. Elke afbeelding wordt als input gegeven aan het netwerk. In de meeste werken worden de gewichten van de laatste laag voor de softmaxlaag gebruikt als representatie van deze afbeelding. \todo{cites hier toevoegen o.a. Chen, Karpathy,Vinyals...} \cite{Xu2015} gebruikt echter ook de lower convolutional layers als extra input.

Bepaalde neurale netwerken maken het ook mogelijk om een afbeelding op te delen in verschillende regio's en voor elke regio een afbeeldingsvector te maken.\cite{Karpathy2015}\cite{Fang}

\section{Caption representaties}
Ook voor de gekende captions kan een representatie nuttig zijn. De meeste modellen met zulke representaties maken gebruik van een vector representatie van elk woord. Vervolgens kan het nodig zijn om deze nog samen te voegen tot een representatie van de volledige zin.

\subsection{Voorstellen van woorden}
 Een voorstelling van woorden als vector vergemakkelijkt enerzijds de verdere verwerking en kan anderzijds ook semantische informatie bieden zoals bij bijvoorbeeld word2vec\cite{Mikolov2013}\todo{cite}
 Een mogelijke eerste voorstelling is een one-hot encodering waarin de plaats waar de vector niet 0 is overeenkomt met het woord. Deze representatie kan verder worden uitgebreid door de vector nog te vermenigvuldigen met een te leren gewichtsmatrix om zo ook woordsemantiek te bevatten. \todo{Het is mogelijk om deze representatie verder uit te breiden met een gewichtmatrix, om zo ook woordsemantiek toe te voegen.} Deze gewichtsmatrix kan willekeurig worden ge\"initialiseerd of eerst worden geleerd op bestaande corpora.\cite{Lebret2013}\cite{Google}\cite{Mao}\todo{cite} Daarna kunnen de gekende woorden en zinnen de gewichten nog verfijnen.  Een andere mogelijkheid is om bestaande word embeddings te gebruiken.\cite{mikolov} Deze hebben als nadeel dat niet voor elk woord uit de captions een vector representatie beschikbaar is. 
 
 \subsection{Voorstellen van zinnen}
 Verschillende mogelijkheden bestaan om de zinnen voor te stellen wanneer de woordvectoren gekend zijn. Een eerste mogelijkheid gebruikt een dependency parser en stelt de zinnen voor als een geparsete afhankelijkheidsboom.\cite{Socher}\todo{fix cite}\cite{Karpathy} gebruikt ook een dependency parser maar probeert hier triplets uit te halen. Een volgende mogelijkheid is om de woordvectoren op te tellen.\cite{Lebret}\todo{cite}Een vaak gebruikt taalmodel in de NLP-literatuur zijn Recurrente Neurale Netwerken (RNN).\cite{Mikolov2010} Dit zijn neurale netwerken die goed overweg kunnen met sequenti\"ele data zoals taal.\cite{Kiros} gebruikt de verborgen lagen van een RNN met als input de zin samen met nog extra informatie over de zin zoals POS-tags. Andere modellen stellen een zin voor als de sequentie van woordvectoren in de zin.
 
\section{Van representaties naar captions}
Verschillende methodes kunnen worden gebruikt om vanuit de representaties van de afbeelding en bijbehorende zinnen een model te trainen dat zo goed mogelijk is in staat is om nieuwe afbeeldingen om te zetten tot zinnige beschrijvingen. De meeste modellen trainen met als doel het verschil tussen de gegenereerde omschrijving en de trainingsafbeelding te minimaliseren.

\subsection{Closest image}
E\'en van de meest eenvoudige aanpakken voor het genereren van een beschrijving bij een ongeziene afbeelding is het zoeken naar de verzameling van de meest gelijkaardige afbeeldingen in de training set. Een gelijkaardigheidsmetriek zoals de cosinusgelijkenis tussen de afbeeldingsrepresentaties biedt hier de oplossing. Het resultaat is dan een lijst met beschrijvingen van de meest gelijkaardige afbeeldingen. Vervolgens cre\"eert het model een rangorde op basis van extra visuele of textuele informatie.\cite{Ordonez2011}\cite{Oliva2006}\cite{Torralba}\cite{Devlin}
Deze modellen hebben als nadeel dat er geen generatie van zinnen die nog niet in de training set zitten, mogelijk is.

Een variatie hiervan \cite{Kuznetsova}\cite{Gupta} zoekt naar de beschrijving van visueel gelijkaardige objecten voor een afbeelding waarbij de objecten in de representatie zitten. Vervolgens detecteert het verschillende soorten phrases (NP,VP,PP) afhankelijk van de gedecteerde objecten en scenes. Met de verzamelde phrases wordt dan een nieuwe zin gegenereerd.

Naast het rechtstreeks gebruiken van de beschrijvingen van de dichstbijzijnde afbeeldingen kunnen deze ook als input worden gebruikt samen met de afbeelding voor een tweede model. \cite{Mason} Zo beschouwt \cite{Mason} bijvoorbeeld het genereren van captions als een samenvattingsprobleem en gebruikt de beschrijvingen van gelijkaardige afbeeldingen als extra input. Daarnaast verkrijgt \cite{Xu} verbeteringen door het toevoegen van extra semantische informatie aan het neurale netwerk, zoals beschrijvingen van gelijkaardige afbeeldingen.
 
\subsection{Multimodale modellen}
Enkele werken proberen een gemeenschappelijke afbeelding-zin embedding te leren zodat het mogelijk wordt om zowel de representatie van zinnen als afbeelding te mappen naar dezelfde ruimte. Dit laat toe om afbeeldingen en zinnen te vergelijken met een afstandsmaat zoals bijvoorbeeld de cosinusgelijkenis. Dit is zeer nuttig voor onder andere image retrieval en sentence retrieval. Het leren van multimodale modellen kan onder andere met Canonical Correlation Analysis (CCA)\cite{Hodosh2013} en neurale netwerken. \cite{Mao2014}\cite{Karpathy2014}\cite{Fang}

\subsection{Template gebaseerd}
Een volgende aanpak baseert zich op templates om zinnen te genereren. Op basis van de waarschijnlijke objecten, scenes, acties, werkwoorden etc. wordt een voorgedefinieerde template uit een lijst templates ingevuld.\cite{Yang} Hiervoor is het dikwijls nodig om bijkomende complexe modellen te trainen zoals bij bijvoorbeeld \cite{Elliott}. Het nadeel van deze methode is dat de gegenereerde zinnen wel syntactisch correct zijn, maar dikwijls onnatuurlijk aanvoelen voor mensen. Om deze methode te verbeteren kunnen gegenereerde of vooraf gekende zinfragmenten helpen bij het recombineren van fragmenten om nieuwe beschrijvingen te genereren. \cite{Mitchell}\cite{Kuznetsova}

\subsection{Neurale netwerken}
De meest recente en best scorende modellen gebruiken echter neurale netwerken voor de generatie van nieuwe zinnen. Deze modellen zijn in staat om compleet nieuwe en voor mensen vlotte zinnen te produceren. Recurrente Neurale Netwerken (RNN) \ref{Mikolov} winnen in de literatuur aan populariteit als taalmodel. RNN's zijn in staat om sequenti\"ele data te genereren op basis van een zekere input. LSTM's (Long Short Term Memory) vormen een uitbreiding op de RNN's en houden informatie bij die ze gedurende een langere termijn kunnen bijhouden in een geheugencel. Beide modellen verwachten een sequentie van woordrepresentaties als input, maar kunnen ook uitgebreid worden met extra informatie. \cite{Kiros}\cite{Xu Kul}\cite{Socher} \todo{kul paper ook ? }

Een eerste verzameling van modellen met neurale netwerken volgen het encoder-decoder principe.\cite{Kiros} De encoder transformeert een afbeelding naar een gemeenschappelijke multimodale ruimte. De decoder transformeert vervolgens deze multimodale representatie naar een zin in natuurlijke taal. Door het multimodale karakter van deze modellen is image en sentence retrieval ook mogelijk. Er bestaan zowel encoder-decoder modellen met LSTM's\cite{Kiros} als met RNN's\cite{Karpathy1}\cite{Mao}.

Een tweede categorie gebruikt de afbeeldingsrepresentatie als extra input naast de sequentie van 
woordrepresenties bij het trainen van het netwerk. Ook hier bestaan er modellen met LSTM \cite{Donahue} en RNN\cite{Karpathy}.

Trainen van het netwerk gebeurt met terugpropagatie doorheen het netwerk. Het is mogelijk om de fouten ook terug te propageren naar de gewichtsvectoren van de woordrepresentaties of naar de gewichten van een CNN.

Genereren van woorden kan gebeuren met sampling of met beam search op de output van het netwerk. Het einde van de zin wordt gekenmerkt met een specifiek stopwoord.

\subsection{Entropie gebaseerde taalmodellen}
Entropie gebaseerde taalmodellen vormen een laatste categorie van modellen. Zo gebruikt \cite{Fang} een statistisch taalmodel in combinatie met een lijst met waarschijnlijke woorden geleerd vanuit de afbeeldingsrepresentatie. Het taalmodel is geleerd op basis van de captions in de training data. In een volgende stap zoeken ze de zinnen die het meest waarschijnlijk zijn gegeven de woorden in de afbeelding. Vervolgens gaan ze de gegeneerde zinnen sorteren op basis van een aantal features. Dit model is net als de modellen met neurale netwerken in staat om nieuwe en vlotte zinnen te vormen. De prestatie is gelijkaardig aan die van de neurale netwerken.

\cite{Lebret} toont aan dat ook met een veel eenvoudiger taalmodel toch redelijk goede resultaten kunnen worden bekomen. Dit model extraheert alle phrases uit de training data en leert daarmee een eenvoudig 3-gram language model. In tegenstelling tot alle voorgaande modellen gebeurt training van de multimodale transformatie met negatieve sampling. Ook hier gebeurt er nog een hersortering.

%%% Local Variables: 
%%% mode: latex
%%% TeX-master: "masterproef"
%%% End: 
\chapter{Theoretische Achtergrond}
\label{hst-theorie}
Dit hoofdstuk bevat de theoretische concepten die nodig zijn om de gebruikte aanpak zo goed mogelijk te begrijpen. Het bevat een beschrijving van de gebruikte neurale netwerken en van een aantal concepten uit de statistiek.

\section{Recurrente Neurale Netwerken}
Recurrente neurale netwerken zijn een uitbreiding van standaard feedforward neurale netwerken. Ze kunnen, net zoals feedforward netwerken, getraind worden met terugpropagatie. Het grote verschil met feedforward netwerken is dat de output van de vorige stap wordt teruggekoppeld naar de verborgen layers. Op figuur \ref{fig:rnn} is te zien hoe een RNN ontrold wordt over de verschillende tijdstippen. Dit zorgt ervoor dat het netwerk in staat is om dingen te onthouden. Hierdoor kunnen recurrente netwerken zeer goed om met het coderen van tijdsgerelateerde informatie. Dit is zeer geschikt om sequenti\"ele data, zoals tekst, te voorspellen. Recurrente neurale netwerken kunnen bijgevolg gebruikt worden als een language model.\

\begin{figure}[tb]
    \centering
    \includegraphics[width=\linewidth]{Images/rnn.PNG}
    \label{fig:rnn}
    \caption{Ontrolling van een recurrent neuraal netwerk}
\end{figure}

Het voorspellen van een zin met een RNN gebeurt woord per woord. Op basis van de eerder waargenomen woorden kan een voorspelling gemaakt worden van het volgende woord. De woorden worden in de vorm van vectorrepresentaties aan het netwerk gegeven. Deze encodering kan gebruik maken van one-hot codering, ze kan random zijn, of er kan gebruik gemaakt worden van word embeddings zoals bijvoorbeeld \emph{word2vec}\cite{Mikolov2013}.
\section{Convolutionele Neurale Netwerken}

\section{Long Short Term Memory Neurale Netwerken}
Long Short Term Memory (LSTM) is een vorm van RNN die geheugencellen bevat. Door deze cellen is het netwerk in staat om op lange termijn informatie over de input bij te houden. Elk LSTM blok heeft een aantal gates om te bepalen of de input moet onthouden worden, en of een vorige waarde moet bijgehouden of vergeten worden. De output van de cellen is bijgevolg afhankelijk van alle eerder geobserveerde inputs. Op figuur \ref{fig:lstm} is te zien hoe een LSTM-blok er uitziet.\cite{Google}\cite{SeppHochreiter1997}

\begin{figure}[tb]
    \centering
    \includegraphics[width=\linewidth]{Images/lstm.PNG}
    \label{fig:lstm}
    \caption{Long Short Term Memory geheugenblok}
\end{figure}

LSTM netwerken worden net als RNN gebruikt als language models en zorgen over het algemeen voor hogere kwaliteit. Dit komt doordat LSTM netwerken een langere periode hebben waarover de input kan onthouden worden. Door het geheugen dat langer kan onthouden dan een RNN is het mogelijk om sequenties met een langere periode tussen belangrijke events te modelleren. 
\section{Latent Dirichlet Allocation}
Latent Dirichlet Allocation is een generatief probabilistisch model voor dsicrete data. Een van de meest gebruikte toepassingen hiervan is het modelleren van een een verdeling van onderwerpen in een set van tekstdocumenten. Dit conecpt is gebaseerd op de veronderstelling dat elk document een zeker kansverdeling heeft over alle mogelijke onderwerpen. Deze onderwerpen hebben dan een kansverdeling over alle mogelijke woorden. Zo kan de kans dat een bepaald document $d_j$ een bepaald woord $w_i$ bevat worden geschreven als een som over alle verschillende topics (formule \ref{formule:lda}). 

\begin{equation}
    \label{formule:lda}
    P(w_i | d_j) = \sum\limits_{k=0}^{n_{topics}}P(w_i|topic_k)P(topic_k|d_j)
\end{equation}

Het generatieve aspect van LDA is te zien in figuur \ref{fig:lda}. Op basis van twee Dirichlet priors $\alpha$ en $\beta$ word een kansverdeling over de onderwerpen gesampled per document ($\theta$), alsook een kansverdeling over de woorden voor elk onderwerp ($\phi$). Uit $\theta$ wordt voor elke positie $i$ in een document $j$ een onderwerp gesampled ($z_{ji}$). Het samplen van de woordverdeling voor dit onderwerp leidt tot het woord $w_{ji}$. Trainen van een LDA model gebeurt bijvoorbeeld met Gibbs sampling, en leidt tot de verdelingen $\theta$ en $\phi$.

\begin{figure}[tb]
    \centering
    \includegraphics[width=\linewidth]{Images/lda.png}
    \label{fig:lda}
    \caption{Grafische weergave van LDA}
\end{figure}

\section{Stacked Canonical Correlation Analysis}

% ... en zo verder tot
\include{hfdst-n}
\chapter{Besluit}
\label{besluit}
Deze masterproef probeert twee bestaande systemen voor afbeeldingsbeschrijving te verbeteren. Dit gebeurt op drie manieren. Ten eerste is er het afleiden extra informatie uit een afbeelding, hetzij door het extraheren van onderwerpen, hetzij door een multimodale projectie. Beide methodes zijn variabel in het aantal gebruikte dimensies. Deze masterproef bestudeert dan ook de invloed van de dimensionaliteit van de extra informatie. 
Een tweede aangebrachte verbetering is het normaliseren van de zinnen tijdens het generatieproces. Dit kan leiden tot zinnen die meer informatie bevatten of tot een betere verdeling van de zinslengtes. Tenslotte bestudeert deze masterproef de mogelijke invloed van een aantal parameters van het generatieproces.

Deze thesis experimenteert ook met de robuustheid van de twee manieren om semantische informatie toe te voegen. Een aantal experimenten proberen de invloed van ruis op deze informatie te bepalen.

Dit hoofdstuk biedt eerst een overzicht van de belangrijkste resultaten en de bekomen verbeteringen. Daarna volgt een overzicht van de mogelijke uitbreidingen.
\section{Resultaten}
\subsection{Toevoegen van semantische informatie}
Het gebruik van extra semantische informatie leidt tot verbeteringen in de kwaliteit van de gegenereerde zinnen. 
Bij RNN zorgt het gebruik van de onderwerpen afgeleid uit de afbeeldingen voor een hogere score ten opzichte van het referentiemodel. 
De veronderstelling dat de ondewerpen de gegenereerde zinnen beter doet aansluiten bij de afbeelding is dus correct.
Bij LSTM geven zowel de afgeleide onderwerpen als de multimodale projectie een hogere score dan het referentiemodel. 
De scores van beide technieken om extra informatie toe te voegen liggen bij LSTM zeer dicht bij elkaar, maar het gebruik van onderwerpen scoort lichtjes beter op de metrieken die het dichtst aanleunen bij menselijke evaluatie.


\subsection{Normalisatie}
Om de lengte van de gegenereerde zinnen beter te doen aansluiten bij die van de trainingsverzameling implementeert deze masterproef een normalisatiefunctie. Hierdoor krijgen zinnen die te hard afwijken een lagere score. 
Deze methode heeft een zeer grote invloed op de gegenereerde zinnen. De gemiddelde lengte stijgt bij RNN van 7 naar 10, en bij LSTM van 8 naar 10. De normalisatiefunctie heeft het meeste invloed op het einde van de zinnen, waardoor het begin van de meeste zinnen vrij generisch is en het grootste deel van de informatie zich in de laatste woorden van de zin bevindt.

Een tweede vorm van normalisatie probeert de gegenereerde zinnen informatiever te maken. Alle woorden uit de woordenschat krijgen een score op basis van hoeveel keer ze voorkomen in de trainingsverzameling. Woorden die minder voorkomen zijn specifieker en krijgen dus een hogere score. Op basis van deze scores krijgen de gegenereerde zinnen een aangepaste score. Deze normalisatie leidt inderdaad tot zinnen die meer specifieke woorden bevatten. Meestal leidt dit ook tot een zin van hogere kwaliteit, maar in een aantal gevallen voegt het systeem schijnbaar willekeurig een aantal woorden met een hoge score toe die weinig met de afbeelding te maken hebben. De evaluatiemetrieken oordelen ook dat de zinnen met deze normalisatie slechter zijn, maar voor een mens zijn een groot aantal van de zinnen van betere kwaliteit.
\subsection{Invloed van parameters}

\subsection{Robuustheid van semantische informatie}

\section{Toekomstig werk}
Zoals beschreven in het literatuuroverzicht en de resultaten scoren de modellen die gebruik maken van aandachtsmechanismes het hoogste. Het gebruik van aandachtsvectoren is echter te complex voor deze masterproef. Het is zeker interessant om te experimenteren met het toevoegen van aandachtsinformatie aan de systemen voorgesteld in deze masterproef om tot grotere verbeteringen te komen. 

Een ander mogelijk vervolg op deze masterproef gaat dieper in op de robuustheid van de verschillende systemen om semantische informatie toe te voegen aan de generatiesystemen.
De aanpak in de experimenten is vrij rudimentair, dus het zou zeker lonen om te onderzoeken wat de invloed is van verschillende gradaties van perturbatie, alsook verschillende types van ruis.
%%% Local Variables: 
%%% mode: latex
%%% TeX-master: "masterproef"
%%% End: 


% Indien er bijlagen zijn:
\appendixpage*          % indien gewenst
\appendix
\chapter{De eerste bijlage}
\label{app:A}
In de bijlagen vindt men de data
% ... en zo verder tot
\include{app-n}

\backmatter
% Na de bijlagen plaatst men nog de bibliografie.
% Je kan de  standaard "abbrv" bibliografiestijl vervangen door een andere.
\bibliographystyle{abbrv}
\bibliography{referenties}

\end{document}

%%% Local Variables: 
%%% mode: latex
%%% TeX-master: t
%%% End: 
